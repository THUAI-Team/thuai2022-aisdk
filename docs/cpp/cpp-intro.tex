\documentclass{article}

\usepackage{ctex}
\usepackage{xcolor}
\usepackage{hyperref}
\usepackage{gitinfo2}
\usepackage{wasysym}
\usepackage{multicol}
\usepackage{footnote}
\usepackage{listings}
\usepackage{hhline}
\usepackage{siunitx}
\usepackage{fontspec}
\usepackage[margin=1.5cm]{geometry}


\lstset{basicstyle=\ttfamily,breaklines=true}

\title{“巧取智夺”赛道C++ SDK}
\author{软院、计算机系联合开发组}
\date{\today\\版本:\gitAbbrevHash}

\begin{document}
\maketitle

\section{简介}

本SDK可以帮助你的AI和评测后端通信。这个SDK由以下文件组成:

\begin{table}[h]
\centering
\begin{tabular}{|l|l|}\hline
	文件名 & 备注\\ \hline
	\texttt{CMakeLists.txt} & \fontshape\scdefault\selectfont\textsc{Cmake}构建系统的配置文件\\ \hline
	\color{red}{\texttt{contestant\_code.cpp}} & 选手代码主文件 \\ \hline
	\texttt{egg\_sdk.\{h,cpp\}} & SDK相关接口的声明,其中通信的具体实现选手无需理会 \\ \hline
	\texttt{schema.\{h,cpp\}} & 对玩家状态、金蛋、坐标等API的声明和实现 \\ \hline
	\texttt{singleton.h} & 单例模式实现 \\ \hline
	\texttt{stream\_helper.\{h,cpp\}} & 通信流辅助函数,选手无需理会\\ \hline
\end{tabular}
\end{table}

\section{环境配置}

为运行SDK,你需要配置以下环境:
	\begin{itemize}\setlength\itemsep{0em}
		\item \textsc{Cmake} 3.10及以上版本
		\item \textsc{Gcc},\textsc{Msvc} 或 \textsc{Clang} 编译器
	\end{itemize}
	对于开发,我们推荐使用 Visual Studio Code 和其 \textsc{Cmake} 插件的组合。你也可以使用 Visual Studio 进行开发,不过需要选择导入 \textsc{Cmake} 项目\footnote{参见 \href{https://docs.microsoft.com/en-us/cpp/build/cmake-projects-in-visual-studio?view=msvc-160}{微软官方文档}}。由于 Dev-C++ 对于 Cmake 的支持并不好,我们并不推荐用它进行开发。

	在下载SDK后,你可以尝试运行 \textsc{Cmake} 进行一次构建,以检查构建环境。如果 Cmake 运行出错,请检查你的本地环境是否正确配置。

\section{开发}

理论上,你只需要修改\texttt{contestant\_code.cpp}这一文件中的 \texttt{update()} 函数。这个函数会在每秒10次\footnote{游戏运行于60fps,每6帧运行一次更新函数,即为每秒运行10次。}的更新中被调用,在其中你可以尝试做出各种动作。请注意:这些函数都不会返回运行的结果,且并不会在调用后立刻体现效果。所有的操作请求都会在 \texttt{update()} 运行结束后一并发送给游戏逻辑。因此,你需要在下一次\texttt{update()}运行时对是否成功执行动作进行检查。
如果你想创建新的源代码文件,有可能需要对 \texttt{CMakeLists.txt} 进行修改,因其默认只包括了当前目录下(不含子目录)的源代码文件。

\begin{table}[t]
\caption{SDK提供的数据结构介绍\label{tab:ds}}
\centering
\begin{tabular}{|l|l||l|l|}\hhline{|--||--|}
\multicolumn{2}{|c||}{PlayerStatus} & \multicolumn{2}{c|}{EggStatus}\\ \hhline{|--||--|}
\texttt{position} & 玩家坐标 & \texttt{position} & 蛋坐标 \\\hhline{|--||--|}
\texttt{facing} & 表示玩家朝向的单位向量 & \texttt{holder} & 拿蛋玩家编号,$-1$表示放在地上\\[0pt]\hhline{|-|-||--|}
\texttt{status} & 玩家运动状态 & \texttt{score} & 蛋的分数 \\\hhline{|-|-|:==:}
\texttt{holding} & 玩家拿的蛋编号,$-1$表示空手 &\multicolumn{2}{c|}{PlayerMovement}  \\ \hhline{:==:|--|}
\multicolumn{2}{|c||}{Team} &\texttt{STOPPED} & 玩家停在原地 \\ \hhline{|--||--|} 
\texttt{RED} & 红队 & \texttt{WALKING} & 玩家正在走路  \\ \hhline{|--||--|}
\texttt{YELLOW} & 黄队 &\texttt{RUNNING} & 玩家正在跑步  \\ \hhline{|--||--|}
\texttt{BLUE} & 蓝队 & \texttt{SLIPPED} & 玩家因碰撞滑倒,本回合操作无效\\ \hhline{|--||--|}\end{tabular}
\end{table}

SDK中提供的主要数据结构见表\ref{tab:ds}。

\subsection{接口}

所有公共接口均位于 GameState 对象中。该对象采用单例模式,代码中可以用 \lstinline{GameState::instance()} 获得实例。

\begin{description}
\item[玩家控制] \lstinline{get_player(Team team_id, int player_id_in_team)}\\[-2pt]
传入队伍和队内玩家编号,获得 PlayerStatus 结构体。若设总的编号为 $x$,则队伍$t$和队内编号$y$由以下公式得出:
\[t = x \div 4, y = x \bmod 4\]
其中 $t=0,1,2$分别对应红、黄、蓝队。

\lstinline{set_status_of_player(int player_id_in_team, PlayerMovement status)}\\[-2pt]
尝试设置自己队伍中玩家的移动状态。如果不满足条件,则设置失败。具体失败的情形为:
\begin{itemize}\setlength\itemsep{0em}
\item 该玩家已经摔倒:此时在站起来(恢复成静止)前不能进行任何操作
\item 抱着蛋时尝试奔跑
\item 体力值不够时尝试奔跑
\end{itemize}

请注意:这些设置类函数\color{red}都没有返回值\color{black},选手必须在下一次调用更新时手动检查是否已设置为指定状态。

\lstinline{set_facing_of_player(int player_id_in_team, Vec2D facing)}\\[-2pt]
传入一个向量和本队玩家编号,设置其朝向(用于走路、奔跑)。注意:若传入的是非单位向量,则会将其变为同向单位向量。传入零向量或者模长过小的向量时,评测逻辑行为未定义。

\lstinline{current_team()}
返回当前AI控制的队伍。

\item[金蛋控制] \lstinline{get_egg(int egg_id)}\\[-2pt]
传入金蛋编号,获得其基本信息。

\lstinline{try_grab_egg(int player_id_in_team, int egg_id)}\\[-2pt]
让队伍中某玩家尝试抓取金蛋。只有满足下列条件时,抓取才能成功:

\begin{itemize}\setlength\itemsep{0em}
\item 蛋在地上且糖豆人中心和蛋表面距离不超过 \SI{0.1}{\meter}(即到蛋中心距离不超过\SI{0.69}{\meter})\footnote{$\diameter_{\text{玩家}}=\SI{0.48}{\meter},\diameter_{\text{金蛋}}=\SI{0.7}{\meter}$}
\item 该蛋由别人拿取,且玩家和蛋距离同样不超过\SI{0.69}{\meter}
\item 多人在同回合抢同一个蛋时,某人和蛋距离最近
\end{itemize}

\lstinline{try_drop_egg(int player_id_in_team, double radian)}\\[-2pt]
让队伍中某玩家尝试放置金蛋。参数中的弧度为以$+x$轴为极轴的极坐标系下,放置蛋相对玩家的方位。蛋在放置后会和玩家刚好相切。
只有满足下列条件时,放置才能成功:

\begin{itemize}\setlength\itemsep{0em}
\item 该玩家手中有蛋
\item 蛋放下后不会卡在他人或其他蛋碰撞箱内
\item 蛋放下后不会卡在墙内
\end{itemize}

\end{description}

\subsection{上交代码}

按照Saiblo的要求,提交 C++ 语言代码只需要上传 \texttt{cpp} 文件夹下的所有文件即可。注意上传文件中,\texttt{CMakeLists.txt} 必须位于压缩包的顶层文件夹。

\end{document}
